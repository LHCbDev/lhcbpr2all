\section{References}
\label{sec:References}

References should be made using Bib\TeX~\cite{BibTeX}. A special style
\texttt{LHCb.bst} has been created to achieve a uniform
style. Independent of the journal the paper is submitted to, the
preprint should be created using this style. Where arXiv numbers
exist, these should be added even for published articles. In the PDF
file, hyperlinks will be created to both the arXiv and the published
version.

\begin{enumerate}

\item Citations are marked using square brackets, and the
  corresponding references should be typeset using Bib\TeX\ and the
  official \lhcb Bib\TeX\ style. An example is in
  Ref.~\cite{Sjostrand:2006za}.

\item For references with four or less authors all of the authors'
  names are listed~\cite{Majorana:1937vz}, otherwise the first author
  is given, followed by \etal. the \lhcb Bib\TeX\ style will
  take care of this.

\item The order of references should be sequential when reading the
  document.

\item The titles of papers should in general be included. To remove
  them, change \texttt{\textbackslash
    setboolean\{articletitles\}\{false\}} to \texttt{true} at the top
  of this template.

\item The obtain the correct bibliographic information, the best
  option is to copy the Bib\TeX\ entry directly from
  \texttt{Inspire}. Some manual editing of the paper titles might be
  required to achieve correct \LaTeX\ formatting.

\item Even if the basic formatting of the Bib\TeX\ entry is taken from
  \texttt{Inspire}, all the data should be cross checked against the
  journal. Often there are minor changes to author initials or
  titles. In case of a difference between the preprint and the
  journal, the bibliographic information from the journal should be
  used.

\item The \texttt{mciteplus}~\cite{mciteplus} package is used in order
  to enable multiple references to show up as a single item in the
  reference list. As an example \texttt{\textbackslash
    cite\{Aaij:2011rj,*Pascoli:2007qh\}} where the \texttt{*}
  indicates that the reference should be merged with the previous
  one. The result of this can be seen in
  Ref.~\cite{Aaij:2011rj,*Pascoli:2007qh}. Be aware that the
  \texttt{mciteplus} package should be included as the very last item
  before the \texttt{\textbackslash begin\{document\}} to work
  correctly.

\item It should be avoided to make references to public notes and
  conference reports in public documents. Exceptions can be discussed
  on a case-by-case basis with the review committee for the
  analysis. In internal reports they are of course welcome and can be
  referenced as seen in~\cite{LHCb-CONF-2011-003} using the
  \texttt{lhcbreport} category. For conference reports, omit the
  author field completely in the Bib\TeX\ record.

\item To get the typesetting and hyperlinks correct for \lhcb reports,
  the category \texttt{lhcbreport} should be used in the Bib\TeX\
  file. See Ref.~\cite{LHCb-INT-2011-047, *LHCb-ANA-2011-078} for some
  examples. It can be used for \lhcb documents in the series
  \texttt{CONF}, \texttt{PAPER}, \texttt{PROC}, \texttt{THESIS}, and
  internal \lhcb reports. Papers sent for publication, but not
  published yet, should be referred with their \texttt{arXiv} number,
  so the \texttt{PAPER} category should only be used in the rare case
  of a forward reference to a paper.

\item Proceedings can be used for references to items such as the
  \lhcb simulation~\cite{LHCb-PROC-2011-006}, where we do not yet have
  a published paper.

\end{enumerate}

There is a set of standard references to be used in \lhcb that are
listed in Appendix~\ref{sec:StandardReferences}.
