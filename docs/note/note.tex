\documentclass{lhcbnote}

\usepackage[utf8]{inputenc}
\usepackage[english]{babel}
\usepackage{graphicx}
\usepackage{tikz}
\usepackage{hyperref}

\title{A New Nightly Build System for LHCb}
\author{Marco Clemencic\address[MCl]{CERN, Switzerland}}
%\ead[eMCl]{marco.clemencic@cern.ch}

\doctyp{Internal Note}
\dociss{0}
\docrev{1}
%\docref{}
\doccre{January 17, 2013}
\docmod{\today}
%\doccnf{}

\begin{document}
\maketitle

\begin{abstract}
The nightly build system used so far by LHCb has been implemented as an
extension on the LCG Application Area one\cite{Kruzelecki:2010zz}.  Although
this version basically works, it has several limitations in terms of
extensibility, management, ease of use, so that the SFT group decided to develop
a new version based on a commercial continuous integration system.

Since, because of technical reasons, we cannot adopt the new planned SFT nightly
build system, we decided to investigate the possibility of a custom version
based on the open source continuous integration system Jenkins\cite{Jenkins}.

In this note I describe the implementation of a working prototype of the new
nightly build system.
\end{abstract}

\begin{status}
\entry{Draft}{1}{January 17, 2013}{First version: introduction and requirements}
%\entry{Draft}{2}{December 16, 2003}{Added conclusion.}
%\entry{Final}{1}{April 26, 2004}{Checked english}
\end{status}

\tableofcontents

\listoffigures
\listoftables

\section{Introduction}
We have been using the Nightly Build System based the the LCG Aplication Area
one for several years\cite{Kruzelecki:2010zz}.  The system works, but it has
limitations that make it very difficult to extend and to manage.  The SFT group
planned a rewrite of their code base, to overcome the limitations sill allowing
for extensions like he ones we needed, so we decided to wait their new
implementation before reviewing our code.  Now that the SFT group opted for an
implementation based on a commercial continuous integration solution that we
cannot easily extend for our use case without a complete rewrite of our code, we
decided to investigate the possibility of a new implementation of the nightly
build system based on the the open source continuous integration system
Jenkins\cite{Jenkins}.

The outcome of the investigation is the working prototype described in this
note.

\section{Requirements}
The new nightly build system must have the main functionalities of the old one,
such as
\begin{itemize}
  \item build and test several slots (consistent set LHCb software projects) on
several platforms
  \item easy configuration of the content of the slots
  \item separate the builds of different platforms
  \item allow customized checkouts (i.e. non default versions of the packages)
  \item run the tests of a project while building the following one on the stack
  \item produce incremental reports of the results of the builds in a dashboard
  \item configurable parsing of the build logs (ignore some warnings and errors)
  \item distribute efficiently the load on a pool of build machines
\end{itemize}
but, wherever possible, improve and simplify the old implementation.

In additions we want to have some long awaited new features:
\begin{itemize}
  \item monitoring of the status of the builds
  \item easy restart at different levels: everything, one slot, one platform of
one slot
  \item produce archives of the checkout and of the builds
  \item easy creation of new slots (both production and testing)
  \item manageable procedure for the development of the system itself
\end{itemize}


%\section{Conclusion}

\bibliography{bibliography}
%\bibliographystyle{plainurl}
\bibliographystyle{unsrturl}

\end{document}
