% $Id: figures.tex 18565 2012-04-26 22:51:51Z uegede $
% ===============================================================================
% Purpose: including figures in the standard template
% Author: Tomasz Skwarnicki, Ulrik Egede
% Created on: 2010-09-24
% ===============================================================================

\section{Figures}
\label{sec:Figures}

A standard \lhcb style file for use in production of figures in \root
is in the \erasmus package \texttt{RootTools/LHCbStyle} or directly in
\svn at
\texttt{svn+ssh://svn.cern.ch/reps/lhcb/Erasmus/trunk/RootTools/LHCbStyle}. It
is not mandatory to use this style, but it makes it easier to follow
the recommendations below.

In Fig.~\ref{fig:example} is shown an example of how to include an eps
or pdf figure with the \texttt{\textbackslash includegraphics} command
(eps figures will not work with \texttt{pdflatex}). Note that if the
graphics sits in \texttt{figs/myfig.pdf}, you can just write
\texttt{\textbackslash includegraphics\{myfig\}} as the \texttt{figs}
subdirectory is searched automatically and the extension \texttt{.pdf}
(\texttt{.eps}) is automatically added for \texttt{pdflatex}
(\texttt{latex}).
\begin{figure}[tb]
  \begin{center}
    \includegraphics[width=0.7\linewidth]{Example1DPlot-python-1}
    \vspace*{-1.0cm}
  \end{center}
  \caption{
    \small %captions should be a little bit smaller than main text
    An example plot using the \lhcb style from the \erasmus package
    \texttt{RootTools/LHCbStyle}. The signal data is shown as points
    with the signal component as yellow (light), background 1 as green
    (medium) and background 2 as blue (dark).}
  \label{fig:example}
\end{figure}

\begin{enumerate}

\item Figures should be legible at the size they will appear in the
  publication, with suitable line width.  Their axes should be
  labelled, and have suitable units (e.g. avoid a mass plot with
  labels in MeV$/c^2$ if the region of interest covers a few GeV$/c^2$
  and all the numbers then run together).  Spurious background shading
  and boxes around text should be avoided.

\item Colour may be used in figures, but the distinction between
  differently coloured areas or lines should be clear also when the
  document is printed in black and white, for example through
  differently dashed lines. The \lhcb style mentioned above implements
  a colour scheme that works well but individual adjustments might be
  required.

\item Figures with more than one part should have the parts labelled
  (a), (b) etc., with a corresponding description in the caption;
  alternatively they should be clearly referred to by their position,
  e.g. Fig.~1\,(left).

\item All figures containing \lhcb data should have \lhcb written on
  them, preferably in the upper right corner.  For preliminary
  results, that should be replaced by ``LHCb preliminary''.

\end{enumerate}
