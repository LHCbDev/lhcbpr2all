\section{Typography}
\label{sec:typography}

The use of the \LaTeX\ typesetting symbols defined in the file
\texttt{lhcb-symbols-def.tex} and detailed in the appendices of this
document is strongly encouraged as it will make it much easier to
follow the recommendation set out below.

\begin{enumerate}

\item \lhcb is typeset with a normal (roman) lowercase b.

\item Titles are in bold face, and usually only the first word is
  capitalized.

\item Mathematical symbols and particle names should also be typeset
  in bold when appearing in titles.

\item Units are in roman type, except for constants such as $c$ or $h$
  that are italic: \gev, \gevcc.  The unit should be separated from
  the value with a space, and they should not be broken over two
  lines.  It is recommended to keep the factors of $c$ for masses and
  momenta, \eg $m = 3.1\gevcc$. However, if they are dropped this
  should be done consistently throughout, and a note should be added
  at the first instance to indicate that units are taken with $c=1$.

\item Italic is preferred for particle names (although roman is
  acceptable, if applied consistently throughout).  Particle Data
  Group conventions should generally be followed: \Bd (no need for a
  ``d'' subscript), \decay{\Bs}{\jpsi\phi}, \Bsb, \KS (note the
  uppercase roman type ``S'').

\item Variables are usually italic: $V$ is a voltage (variable), while
  1 V is a volt (unit).

\item Subscripts are roman type when they refer to a word (such as T
  for transverse) and italic when they refer to a variable (such as
  $t$ for time): \pt, \dms, $t_{\rm rec}$.

\item Standard function names are in roman type: \eg $\cos$, $\sin$
  and $\exp$.

\item Figure, Section, Equation, Chapter and Reference should be
  abbreviated as Fig., Sect. (or alternatively Sec.), Eq., Chap.\ and
  Ref.\ respectively, when they refer to a particular (numbered) item,
  except when they start a sentence. Table and Appendix are not
  abbreviated.  The plural form of abbreviation keeps the point after
  the s, \eg Figs.~1 and~2.

\item Common abbreviations derived from Latin such as ``for example''
  (\eg), ``in other words'' (\ie), ``and so forth'' (\etc), ``and
  others'' (\etal), ``versus'' (\vs) can be used, with the typography
  shown; other more esoteric abbreviations should be avoided.

\item Units, material and particle names are usually lower case if
  spelled out, but often capitalized if abbreviated: amps (A), gauss
  (G), lead (Pb), silicon (Si), kaon (\kaon), proton (\proton).

\item The prefix for 1000 (k, \eg kV) should not be confused with
  that used in computing (K, which strictly speaking denotes $2^{10}$,
  \eg KB).

\item Counting numbers are usually written in words if they start a
  sentence, or if they have a value of ten or below in descriptive
  text (\ie\ not including figure numbers such as ``Fig.\ 4'', or
  values followed by a unit such as ``4\,cm''). Numbers should not be
  written as words if they by nature are real numbers that happen to
  take an integer value, such as $\chisq/\rm{ndf} < 4$.

\item Numbers larger than 9999 have a comma (or a small space) between
  the multiples of thousand: \eg 10,000 or 12,345,678.  The decimal
  point is indicated with a point rather than a comma: \eg 3.141.

\item We apply the rounding rules of the
  PDG~\cite{PDG2012}. The basic rule states that if the three
  highest order digits of the error lie between 100 and 354, we round
  to two significant digits. If they lie between 355 and 949, we round
  to one significant digit. Finally, if they lie between 950 and 999,
  we round up to 1000 and keep two signifcant digits. In all cases,
  the central value is given with a precision that matches that of the
  error. So, for example, the result $0.827 \pm 0.119$ should be
  written as $0.83\pm 0.12$, while $0.827\pm 0.367$ should turn into
  $0.8\pm 0.4$. When writing numbers with error components from
  different sources, \ie statistical and systematic errors, the rule
  applies to the error with the best precision, so $0.827\pm
  0.367\stat\pm 0.179\syst$ goes to $0.83\pm 0.37\stat\pm 0.18\syst$ and
  $8.943\pm 0.123\stat\pm 0.995\syst$ goes to $8.94\pm 0.12\stat\pm
  1.00\syst$.

\item When rounding numbers in a table, some variation of the rounding
  rule above may be required to achieve uniformity.

\item When rounding numbers, it should be avoided to pad with zeroes
  at the end. So $5123 \pm 456$ should be rounded as $(5.12 \pm 0.46)
  \times 10^3$ and not $5120 \pm 460$.

\item Hyphenation should be used where necessary to avoid ambiguity,
  but not excessively. For example: ``big-toothed fish'', but ``big
  white fish''.  Cross-section is hyphenated.

\item Minus signs should be in a proper font ($-1$), not just hyphens
  (-1); this applies to figure labels as well as the body of the text.

\item Inverted commas (around a title, for example) should be a
  matching set of left- and right-handed pairs: ``Title''. The use of
  these should be avoided where possible.

\item Single symbols are preferred for variables in equations, \eg\
  \BF\ rather than BF for a branching fraction.

\item Parentheses are not usually required around a value and its
  uncertainty, before the unit, unless there is possible ambiguity: so
  $\dms = 20 \pm 2\invps$ does not need parentheses, whereas $f_d =
  (40 \pm 4)$\% does.  The unit should not need to be repeated in
  expressions like $1.2 < E < 2.4\gev$.

\item The same number of decimal places should be given for all values
  in any one expression (\eg $5.20 < m_B < 5.34\gevcc$).

\item Apostrophes are best avoided for abbreviations: if the abbreviated term
  is capitalized or otherwise easily identified then the plural can simply add
  an s, otherwise it is best to rephrase: \eg HPDs, \pizs, pions, rather
  than HPD's, \piz's, $\pion$s.

\end{enumerate}
